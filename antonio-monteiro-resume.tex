%
% LaTeX source of my resume
% =========================
%
% Heavily commented to to fit even LaTeX beginners (hopefully).
%
% See the `README.md` file for more info.
%
% This file is licensed under the CC-NC-ND Creative Commons license.
%


% Start a document with the here given default font size and paper size.
\documentclass[10pt,a4paper]{article}

% Set the page margins.
\usepackage[a4paper,margin=0.75in]{geometry}

% Setup the language.
\usepackage[english]{babel}
\hyphenation{Some-long-word}

% Makes resume-specific commands available.
\usepackage{resume}




\begin{document}  % begin the content of the document
\sloppy  % this to relax whitespacing in favour of straight margins


% title on top of the document
\maintitle{António Nuno Monteiro}{August 13, 1990}{Last update on \today}

\nobreakvspace{0.3em}  % add some page break averse vertical spacing

% \noindent prevents paragraph's first lines from indenting
% \mbox is used to obfuscate the email address
% \sbull is a spaced bullet
% \href well..
% \\ breaks the line into a new paragraph
\noindent\href{mailto:anmonteiro.at.gmail.dot.com}{anmonteiro\mbox{}@\mbox{}gmail.com}\sbull
\textsmaller{+}351.937557510\sbull
{\newnums \href{http://www.github.com/anmonteiro}{anmonteiro}} \emph{(GitHub)}\sbull
\href{http://www.linkedin.com/in/anmonteiro}{anmonteiro} \emph{(LinkedIn)}
\\
Vale Grou de Cima, 4\sbull
2815-658\sbull
%2815-658\thinspace {\large \sc ld}\sbull
Sobreda de Caparica\sbull
Portugal

\spacedhrule{0.9em}{-0.4em}  % a horizontal line with some vertical spacing before and after

\roottitle{Summary}  % a root section title

\vspace{-1.3em}  % some vertical spacing
\begin{multicols}{2}  % open a multicolumn environment
\emph{Curious by nature and a geek \emph{per se}, open source enthusiast with an entrepreneurial mindset, a passion for continuous learning and delivering value by designing and developing well-tested, maintainable software.}\\

After being introduced to programming as a freshman Electrical Engineering student, António quickly realised that he wanted to pursue a Software Development and Engineering career, dropping out to study Computer Science and Engineering. From 2010 to 2014, he studied and graduated in Computer Science, but didn't settle for what was being taught in classes, teaching himself a variety of skills outside the scope of the \emph{curriculum}, such as Python and Bash programming, \LaTeX\ and Beamer typesetting.

His passion for \acr{UNIX} and the open source movement made him want to try quite some different flavours of Linux, finally settling for \href{https://www.archlinux.org/}{ArchLinux} until, more recently, switching to OSX.

His curious by nature mindset took him to acknowledge the benefits of \acr{TDD}/\acr{BDD} and Extreme Programming methodologies, becoming almost an evangelist of Test Driven Development and Continuous Integration best practices.

His most recent passion is Full-Stack JavaScript and Node.js development, as well as document oriented NoSQL databases, which he is always eager to learn more about and build applications with.
\end{multicols}


\spacedhrule{0em}{-0.4em}

\roottitle{Experience}

\headedsection  % sets the header for the section and includes any subsections
  {\href{http://www.safira.pt}{Safira}}
  {\textsc{Lisbon, Portugal}} {%
  \headedsubsection
    {Software Developer}
    {Sep \apo13 -- present}
    {\bodytext{Safira is a Portuguese IT consulting boutique focused on improving the way organisations do business, mainly through \href{http://www-03.ibm.com/software/products/en/business-process-manager-family}{IBM BPM} applications. In charge of designing and developing IBM BPM process applications to the company's clients. Delivering consistent, production ready JavaScript (mostly plain JavaScript -- on the server --, jQuery and Dojo -- on the client side), Java and SQL code. Administering WebSphere servers and clusters. Analysing and delivering performance optimisations on the applications and the WebSphere JVM.}}
}

\headedsection  % sets the header for the section and includes any subsections
  {\href{http://www.cmvm.pt}{CMVM - Comissão do Mercado de Valores Mobiliários}}
  {\textsc{Lisbon, Portugal}} {
  \headedsubsection
    {Summer Intern}
    {Jul \apo13 -- Sep \apo13}
    {\bodytext{CMVM, the Portuguese Securities Market Commission, is responsible for supervising and regulating securities and other financial instruments markets (traditionally known as ``stock markets``) in Portugal. The summer internship was focused on helping implementing the \href{http://www.itil-officialsite.com/}{ITIL} standard within the IT department of the organisation.}}
}

\vspace{-0.2em}
\begin{center}
  \emph{\small Please refer to my \href{http://www.linkedin.com/in/anmonteiro}{LinkedIn profile} for a more complete list of work experiences along with recommendations.}
\end{center}


\spacedhrule{-0.2em}{-0.4em}

\roottitle{Education}

\headedsection
  {\href{http://www.fct.unl.pt}{Faculdade de Ciências e Tecnologia - UNL}}
  {\textsc{Caparica, Portugal}} {
  \headedsubsection
    {Bachelor's degree in Computer Science \& Engineering}
    {2010 -- 2014}
    {\bodytext{Finished with a grade point average of 17/20. Focused on algorithms and data structures, distributed systems and networks, and databases. Picked up quite some Java skills in the process.}}
}

\headedsection
  {\href{http://www.esfmp.pt}{Escola Secundária Fernão Mendes Pinto}}
  {\textsc{Almada, Portugal}} {
  \headedsubsection
    {Science and Technology \textnormal{~(pre-university secondary education)}}
    {2002 -- 2008}
    {\bodytext{Finished with a grade point average of 18/20}}
}

\spacedhrule{0.5em}{-0.4em}

\roottitle{Other Education / Certifications:}
\headedsection
  {\href{https://university.mongodb.com}{MongoDB University}}
  {\textsc{Online}} {
  \headedsubsection
    {M101JS: MongoDB for Node.js Developers (\href{http://education.mongodb.com/downloads/certificates/b5bebce320c047bda51f89482ff28948/Certificate.pdf}{Certificate})}
    {2014}
    {}
}
\headedsection
  {\href{http://www.cambridgeenglish.org/}{Cambridge University}}
  {\textsc{Lisbon, Portugal}} {
  \headedsubsection
    {Certificate of Proficiency in English (\acr{CPE})}
    {2011}
    {\bodytext{Grade B.}}
}
\headedsection
  {\href{http://www.academiaam.com/}{Academia de Amadores de Música}}
  {\textsc{Lisbon, Portugal}} {
  \headedsubsection
    {Musical Theory / Piano / Analysis and Techniques of Composition}
    {1999 -- 2008}
    {\bodytext{Completed the ABRSM equivalent of Musical Theory grade 8 and Piano grade 5. Finished the 3-year course of Analysis and Techniques of Composition, which focuses on a general understanding of the musical composition processes and techniques spanning from the 13th to the 21st century.}}
}

\spacedhrule{0.5em}{-0.4em}

\roottitle{Skills}
\inlineheadsection  % special section that has an inline header with a 'hanging' paragraph
  {Technical expertise:}
  {Software design and development, with(in) a team.  Big fan of Agile and XP (Extreme Programming) methodologies, TDD (BDD).  Loves writing JavaScript/\nsp Java/\nsp Python, and has been appreciating functional languages such as Lisps (particularly Clojure/\nsp ClojureScript) further and further.  Strong understanding of relational and document oriented databases; substantial technical skills in \acr{SQL}/\nsp \acr{T-SQL}.  Solid knowledge of web technologies:\ \acr{HTML5}, \acr{CSS3}, \acr{XML}, \acr{REST}, \acr{SOAP}, \acr{JSON}, \acr{AJAX}/\nsp\acr{XHR} and JavaScript (mostly ECMAScript 5.1, Node.js and jQuery). Fan of Git as a distributed version control system.}

\vspace{0.5em}
\inlineheadsection
  {Natural languages:}
  {Portuguese \emph{(mother tongue)}, English \emph{(native or bilingual proficiency)}, French \emph{(limited working proficiency)}.}


\spacedhrule{1.6em}{-0.4em}

\roottitle{Interests}

\inlineheadsection
  {Non-exhaustive and in alphabetical order:}
  {bossanova (music), databases, functional programming, guitar, JavaScript, history, literature, music, open source, piano, programming, programming languages, programming paradigms, software engineering (best practices and methodologies), travel, \acr{UI}/\acr{UX}-design.}

\end{document}
