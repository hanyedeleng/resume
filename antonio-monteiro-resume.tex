%
% LaTeX source of my resume
% =========================
%
% Heavily commented to to fit even LaTeX beginners (hopefully).
%
% See the `README.md` file for more info.
%
% This file is licensed under the CC-NC-ND Creative Commons license.
%


% Start a document with the here given default font size and paper size.
\documentclass[10pt,a4paper]{article}

% Set the page margins.
\usepackage[a4paper,margin=0.75in]{geometry}

% Setup the language.
\usepackage[english]{babel}
\hyphenation{Some-long-word}

% Makes resume-specific commands available.
\usepackage{resume}


\begin{document}  % begin the content of the document
\sloppy  % this to relax whitespacing in favour of straight margins

% title on top of the document
\maintitle{António Nuno Monteiro}{August 13, 1990}{Last update on \today}

\nobreakvspace{0.3em}  % add some page break averse vertical spacing

% \noindent prevents paragraph's first lines from indenting
% \mbox is used to obfuscate the email address
% \sbull is a spaced bullet
% \href well..
% \\ breaks the line into a new paragraph
\noindent\href{mailto:anmonteiro.at.gmail.dot.com}{anmonteiro\mbox{}@\mbox{}gmail.com}\sbull
\textsmaller{+}49.17645619843\sbull
{\newnums \href{http://www.github.com/anmonteiro}{anmonteiro}} \emph{(GitHub)}\sbull
\href{http://www.linkedin.com/in/anmonteiro}{anmonteiro} \emph{(LinkedIn)}
\\
Hochschulstr. 50 / 1510\sbull
01069 Dresden\sbull
%2815-658\thinspace {\large \sc ld}\sbull
Deutschland

\spacedhrule{0.9em}{-0.4em}  % a horizontal line with some vertical spacing before and after

\roottitle{Summary}  % a root section title

\vspace{-1.3em}  % some vertical spacing
\begin{multicols}{2}  % open a multicolumn environment
\emph{Curious by nature, António is an open source enthusiast with an entrepreneurial
mind and a passion for continuous learning, software engineering and delivering value
by designing and developing well-tested, maintainable software.} \\

António is sofware engineer, speaker, and a finalist of the International Master's
Program in Distributed Systems Engineering at the Technical University of Dresden,
Germany. His interests span from topics such as programming language design, distributed
systems and databases, to software architecture, user interfaces and user experience.
He is passionate about functional programming, and has been specializing in the Clojure
programming language. António is a regular contributor to projects such as the
\href{http://clojurescript.org/}{ClojureScript} compiler and \href{https://github.com/omcljs/om}{Om},
a library for building user interfaces based on the latest developments in industry.
\end{multicols}


\spacedhrule{0em}{-0.4em}

\roottitle{Experience}

\headedsection  % sets the header for the section and includes any subsections
  {\href{http://se.inf.tu-dresden.de}{TU Dresden - Fakultät Informatik - Systems Engineering Group}}
  {\textsc{Dresden, Deutschland}} {%
  \headedsubsection
    {Studentische Hilfskraft}
    {Feb \apo15 -- Present}
    {\bodytext{Student Research Assistant at the Systems Engineering Group of TU Dresden. Tasks include development and analysis of distributed systems such as Zookeeper and client implementations in Python.}}
}

\headedsection  % sets the header for the section and includes any subsections
  {\href{http://www.safira.pt}{Safira}}
  {\textsc{Lisbon, Portugal}} {%
  \headedsubsection
    {Software Developer}
    {Sep \apo13 -- Sep \apo14}
    {\bodytext{Safira is a Portuguese IT consulting boutique focused on improving the way organisations do business, mainly through \href{http://www-03.ibm.com/software/products/en/business-process-manager-family}{IBM BPM} applications. In charge of designing and developing IBM BPM process applications to the company's clients. Delivering consistent, production ready JavaScript (mostly plain JavaScript -- on the server --, jQuery and Dojo -- on the client side), Java and SQL code. Administering WebSphere servers and clusters. Analysing and delivering performance optimisations on the applications and the WebSphere JVM.}}
}

\headedsection  % sets the header for the section and includes any subsections
  {\href{http://www.cmvm.pt}{CMVM - Comissão do Mercado de Valores Mobiliários}}
  {\textsc{Lisbon, Portugal}} {
  \headedsubsection
    {Summer Intern}
    {Jul \apo13 -- Sep \apo13}
    {\bodytext{CMVM, the Portuguese Securities Market Commission, is responsible for supervising and regulating securities and other financial instruments markets (traditionally known as ``stock markets``) in Portugal. The summer internship was focused on helping implementing the \href{http://www.itil-officialsite.com/}{ITIL} standard within the IT department of the organisation.}}
}

\vspace{-0.2em}
\begin{center}
  \emph{\small Please refer to my \href{http://www.linkedin.com/in/anmonteiro}{LinkedIn profile} for a more complete list of work experiences along with recommendations.}
\end{center}


\spacedhrule{-0.2em}{-0.4em}

\roottitle{Education}

\headedsection
  {\href{http://www.tu-dresden.de}{Technische Universität Dresden}}
  {\textsc{Dresden, Deutschland}} {
  \headedsubsection
    {MSc -- Distributed Systems Engineering}
    {2014 -- present}
    {\bodytext{Focused on systems engineering, fault tolerance and scalability, distributed systems and its applications and software engineering.}}
}

\headedsection
  {\href{http://www.fct.unl.pt}{Faculdade de Ciências e Tecnologia - UNL}}
  {\textsc{Caparica, Portugal}} {
  \headedsubsection
    {BSc -- Computer Science \& Engineering}
    {2010 -- 2014}
    {\bodytext{Finished with a grade point average of 16/20. Focused on algorithms and data structures, distributed systems and networks, and databases. Picked up quite some Java skills in the process.}}
}

\headedsection
  {\href{http://www.esfmp.pt}{Escola Secundária Fernão Mendes Pinto}}
  {\textsc{Almada, Portugal}} {
  \headedsubsection
    {Science and Technology \textnormal{~(pre-university secondary education)}}
    {2002 -- 2008}
    {\bodytext{Finished with a grade point average of 18/20}}
}

\spacedhrule{0.5em}{-0.4em}

\roottitle{Selected talks:}
\headedsection
  {\href{http://www.ustream.tv/recorded/86179814}{Clients in control: building demand-driven systems with Om Next}}
  {\textsc{Budapest, Hungary}} {
  \headedsubsection
    {Craft Conference}
    {April 2016}
    {}
}
\headedsection
  {\href{https://www.youtube.com/watch?v=Zb18iPjDgwM}{Clients in control: building demand-driven systems with Om Next}}
  {\textsc{Barcelona, Spain}} {
  \headedsubsection
    {Full Stack Fest}
    {September 2016}
    {\bodytext{Traditional architectures are no longer suitable for the increasing needs of today's applications. The price is often paid in high bandwidth and reduced performance. Demand-driven design enables clients to request arbitrary data on demand. Companies like Facebook and Netflix have switched to demand-driven architectures to better embrace a great variety of continuously changing clients. Solutions like Relay and Falcor/JSONGraph distill such ideas. Om Next builds on, and extends these concepts further, to provide a Clojure(Script) based solution. In this talk, I present the motivation for a demand-driven approach and explore the benefits and tradeoffs that Om Next brings to the table.}
}}

\spacedhrule{0.5em}{-0.4em}

\roottitle{Other Education / Certifications:}
\headedsection
  {\href{https://university.mongodb.com}{MongoDB University}}
  {\textsc{Online}} {
  \headedsubsection
    {M101JS: MongoDB for Node.js Developers (\href{http://education.mongodb.com/downloads/certificates/b5bebce320c047bda51f89482ff28948/Certificate.pdf}{Certificate})}
    {2014}
    {}
}
\headedsection
  {\href{http://www.cambridgeenglish.org/}{Cambridge University}}
  {\textsc{Lisbon, Portugal}} {
  \headedsubsection
    {Certificate of Proficiency in English (\acr{CPE})}
    {2011}
    {\bodytext{Grade B.}}
}
\headedsection
  {\href{http://www.academiaam.com/}{Academia de Amadores de Música}}
  {\textsc{Lisbon, Portugal}} {
  \headedsubsection
    {Musical Theory / Piano / Analysis and Techniques of Composition}
    {1999 -- 2008}
    {\bodytext{Completed the ABRSM equivalent of Musical Theory grade 8 and Piano grade 5. Finished the 3-year course of Analysis and Techniques of Composition, which focuses on a general understanding of the musical composition processes and techniques spanning from the 13th to the 21st century.}}
}

\spacedhrule{0.5em}{-0.4em}

\roottitle{Skills}
\inlineheadsection  % special section that has an inline header with a 'hanging' paragraph
  {Technical expertise:}
  {Software design and development, with(in) a team. Loves writing Clojure(Script)/\nsp
JavaScript/\nsp Java/\nsp Python/\nsp Go, and has a passion for functional languages
such as Lisps (particularly Clojure(Script)). Strong understanding of relational
and document oriented databases; substantial technical skills in \acr{SQL}/\nsp
\acr{T-SQL}. Solid knowledge of web technologies; fan of Git as a distributed version
control system.}

\vspace{0.5em}
\inlineheadsection
  {Natural languages:}
  {Portuguese \emph{(mother tongue)}, English \emph{(native or bilingual proficiency)},
French \emph{(limited working proficiency)}, German \emph{(limited working proficiency)}.}

\spacedhrule{1.6em}{-0.4em}

\roottitle{Interests}

\inlineheadsection
  {Non-exhaustive and in alphabetical order:}
  {bossanova (music), databases, design patterns (software), distributed systems,
functional programming, guitar, JavaScript, history, LISP, literature, music, open
source, piano, programming, programming languages, programming paradigms, software
engineering (best practices and methodologies), travel, \acr{UI}/\acr{UX}-design.}

\end{document}
